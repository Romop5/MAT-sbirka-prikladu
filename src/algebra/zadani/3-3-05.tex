Uvažujeme algebru $A=\left ( \Sigma ^{*},\mu , \delta_{a},b \right )$ typu
$(3,1,0)$, kde $\Sigma^{*}$ je množina všech konečných řetězců (slov)
vytvořených z prvků (písmen) konečné množiny (abecedy) $\Sigma$. Symbol $\mu$
označuje ternární operaci zřetězení tří slov v daném pořadí, nulární operace $b$
je dána vybraným prvkem $b \in \Sigma$, $a \in \Sigma$ je pevně daný prvek $a
\neq b$ a $\delta_{a}$ je unární operace, která nahrazuje všechny výskyty prvku
$b$ v daném řetězci řetězce $ab$. Definujme binární relaci $\sim$ na
$\Sigma^{*}$ takto: $u \sim v \Leftrightarrow \left | u \right |=\left | v
\right |$, kde $\left | u \right |$ je počet prvků řetězce $u$. Rozhodněte, zda
$\sim$ je kongruencí na algebře $A$, a pokud ano, popište třídy příslušného
rozkladu. Pokud ne, pak najděte takovou podalgebru algebry $A$, pro kterou
příslušné zúžení  relace $\sim$ kongruencí je.
