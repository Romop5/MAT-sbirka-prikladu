\subsubsection{}
Ve vektorovém prostoru $\mathbb{R}_{3}$ s euklidovskou metrikou $p$ definujeme
vzdálenost libovolných dvou množin $A$ a $B$ vztahem $\delta(A,B) = inf \left \{  (p(a,b)|a \in A, b \in B )\right \}$. Rozhodněte, zda $(P(\mathbb{R}_{3}), \delta)$
tvoří metrický prostor (symbol $P(\mathbb{R}_{3})$ značí množinu všech podmnožin
množiny $\mathbb{R}_{3}$).
\subsubsection{}
Na $\mathbb{Z}^{2}$ definujeme metriku $\delta$ následovně: $\delta
((x_{1},y_{1}),(x_{2},y_{2}))=\left | x_{1}-x_{2} \right |+\left | y_{1}-y_{2}
\right |$. Zakreslete kružnici určenou touto metrikou a poloměru 2 se středem v
bodě $(0,0)$, tj. množinu 
$$S_{\delta}(2) = \left \{ (x,y) \in \mathbb{Z}^{2}:
\delta((x,y),(0,0))=2 \right \}$$. 
Určete počet prvků množiny $S_{\delta}(2)$ a tyto prvky vypiště.
