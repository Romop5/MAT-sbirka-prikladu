\subsubsection{}
Definujeme zobrazení $\delta : \mathbb{R}^{2}\rightarrow \mathbb{R}$ předpisem
$$\delta ((x_{1},y_{1}),(x_{2},y_{2}))=\frac{\left | x_{1}-x_{2} \right
|}{2}+3\left | y_{1}-y_{2} \right |$$
Rozhodněte, zda zobrazení $\delta$ definuje metriku na množině $\mathbb{R}^2$
(využijte skutečnost, že vztahem $d(x,y) = \left | x-y \right |$ je definována
metrika na $\mathbb{R}$). V kladném případě zakreslete v rovině $\mathbb{R}^2$
jednotkovou kružnici vzhledem k této metrice, tj. množinu $\left \{ \left ( x,y
\right ) \in \mathbb{R}^{2};\delta((x,y),(0,0))=1\right \}$.
