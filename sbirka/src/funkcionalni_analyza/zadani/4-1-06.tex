Nad abecedou $\Gamma =\left \{ x,y,z \right \}$ uvažujeme jazyk
$\Sigma=x^{*}y^{+}z^{*}$. Buď $\mu (u,v)=n$, kde $n$ je nejmenší počet změn
řetězce $u$, které je potřeba provést, aby se tento řetězec transformoval na
řetězec $v$. Přitom změnou řetězce rozumíme vypuštění či vložení symbolu nebo
nahrazení symbolu jiným symbolem v tomto řetězci. Ověřte (dokažte), zda $\mu$ je
či není metrika na $\Sigma$ a v kladném případě určete všechny prvky množiny
$\Sigma$, které leží v otevřené kouli o poloměru 2 se středem v prvku $xyz$.
