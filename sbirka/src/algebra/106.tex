\subsubsection{}
Uvažujme univerzální algebru 
$A=\left ( \mathbb{Z}^{2},e,\delta,\oplus ,\odot, \nabla \right )$, 
kde $e$ je nulární operace, $\delta$ je unární
operace, $\oplus, \odot$ jsou binární operace a $\nabla$ je ternární
operace. Tyto operace jsou dány následovně: $e=(0,1)$, $\delta(x,y)=(x,y+2)$,
$\oplus(x_{2},y_{2})=(x_{1} + x_{2}, y_{1}+y_{2})$, $(x_{1},y_{1}) \odot
(x_{2},y_{2})=(x_{1}x_{2},y_{1}+y_{2})$, $\nabla ((x_{1},y_{1}),
(x_{2},y_{2}), (x_{3},y_{3})) = (x_{1} + x_{2} + x_{3}, y_{1} + y_{2} + y_{3})$.
Zjistěte a zdůvodněte, zda zobrazení $\varphi : \mathbb{Z} \rightarrow
\mathbb{Z}$ určené předpisem $\varphi(x,y) = (3x, x+y)$ je homomorfismus algebry
$A$ do $A$.
