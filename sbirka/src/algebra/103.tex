\subsubsection{}
Nechť pro libovolné přirozené číslo $m>0$ značí symbol $Z_{m}$ okruh zbytkových
tříd modulo $m$ a pro libovolné $x \in Z$ nechť symbol $\left [ x \right ]_{m}$
značí tu třídu kongruence modulo $m$ (tedy prvek množiny $Z_{m}$), která
obsahuje prvek $x$. Jaký musí být vztak mezi přirozenými čisly $m,n>0$, aby
platilo $\left [ x \right ]_{m} \subseteq \left [ x \right ]_{n}$ pro všechna 
$z \in Z$? Je pak zobrazení $f:Z_{m} \rightarrow Z_{n}$ dané předpisem $f(\left [ x
\right ]_{m}) =\left [ x \right ]_{n}$ pro všechna $x \in Z$ homomorfismus?
