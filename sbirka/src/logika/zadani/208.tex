\subsubsection{}
Uvažujme jazyk L s rovností, jedním unárním predikátovým symbolem $p$ a jedním
binárním funkčním symbolem $f$. Nechť $M$ je taková realizace jazyka L na
množině $P(\mathbb{R}^{2})$ všech podmnožin reálné roviny $\mathbb{R}^{2}$, kde
$p_{M}(X)$ znamená, že $X$ je neprázdná množina bodů ležících uvnitř a na
hranici nějakého obdelníku v $\mathbb{R}^{2}$, jehož strany jsou rovnoběžné se
souřadnými osami, $f_{M}(X,Y) = X \cap Y$. Rozhodněte a zdůvodněte, zda
\begin{enumerate}[(1)]
  \item $M \models (\exists x)(f(x,x)=x \Rightarrow p(x))$
  \item $M \models (p(x) \wedge p(y)) \Rightarrow p(f(x,y))$
  \item $(p(x) \wedge p(y)) \models p(f(x,y))$
\end{enumerate}

