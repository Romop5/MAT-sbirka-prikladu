\subsubsection{}
Uvažujme jazyk L se dvěma konstantami $k$, $l$, jedním unárním funkčním symbolem
$f$ a jedním binárním predikátovým symbolem $p$. Nechť $\Re$ je realizace jazyka
L, kde univerzem je množina všech bodů kulové plochy $K$ se středem $O$ s
kulovou plochou $K$. Symbol $f$ se realizuje v bodě $x$ jako bod jemu
protilehlý, tj. $f_{\Re}(x) \neq x$ je průsečík přímky procházející bodem  $x$ a
středem $O$ s kulovou plochou $K$. Realizace konstant jsou dva vzájemně
protilehlé body: $k_{\Re} = S$ (severní pól) a $l_{\Re} = J$ (Jížní pól).
Realizace symbolu $p$ na bodech $x,y$ je $p_{\Re}(x,y) \Leftrightarrow x,y$
leží na stejné (zeměpisné) rovnoběžce, tj. kružnicí vzniklé průnikem kulové
plochy $K$ a roviny kolmé na spojnici bodů $S$ a $J$. Uvažujme následující
formule:
\begin{enumerate}[(1)]
  \item $\chi: p(x, f(x))$
  \item $\psi: p(l,x) \Leftrightarrow p(k,x)$
  \item $\theta: f(k) = l$
\end{enumerate}
Určete ty z teorií $A=\left \{ \psi, \theta \right \}$, $B=\left \{
\neg\chi,\psi \right \}$, $C=\left \{ \neg\chi, \theta \right \}$, $D= \left \{
\psi, \theta \right \}$, jejichž je $\Re$ modelem.
